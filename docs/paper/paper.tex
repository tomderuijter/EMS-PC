% Needed packages
\documentclass[a4paper, 10pt, english, twocolumn]{article}
\usepackage[english]{babel}
\usepackage[cm]{fullpage}
\usepackage{cite}
\usepackage{anysize}
\usepackage[compact]{titlesec}
\usepackage{graphicx}
\usepackage{stfloats}
\usepackage{listings}
\usepackage{hyperref}

\usepackage{amssymb,amsmath}
\usepackage{algorithmicx}
\usepackage{algorithm}
\usepackage[noend]{algpseudocode}

% for coloring individual cells in a table
\usepackage[table]{xcolor}
%\usepackage{pgfgantt}

\newcommand{\keywords}[1]{\par\noindent 
{\bf Keywords\/}. #1}

% Margins & Headers
\marginsize{2.5cm}{2.5cm}{3.0cm}{2.0cm}
\columnsep 0.4in
\footskip 0.4in 
\usepackage{changepage}

% E-mail formatting
\usepackage{color,hyperref}
    \catcode`\_=11\relax
    \newcommand\email[1]{\_email #1\q_nil}
    \def\_email#1@#2\q_nil{
      \href{mailto:#1@#2}{{\emailfont #1\emailampersat #2}}
    }
    \newcommand\emailfont{\sffamily}
    \newcommand\emailampersat{{\color{red}\small@}}
    \catcode`\_=8\relax 
	
% List modifications
\newenvironment{packed_item}{
\begin{itemize}
  \setlength{\itemsep}{1pt}
  \setlength{\parskip}{0pt}
  \setlength{\parsep}{0pt}
}{\end{itemize}}

\newenvironment{packed_enum}{
\begin{enumerate}
  \setlength{\itemsep}{1pt}
  \setlength{\parskip}{0pt}
  \setlength{\parsep}{0pt}
}{\end{enumerate}}

% ### Mathematics ###
\newcommand{\bpm}{\begin{pmatrix}}
\newcommand{\epm}{\end{pmatrix}}

\newcommand{\bbm}{\begin{bmatrix}}
\newcommand{\ebm}{\end{bmatrix}}

\newcommand{\bsm}{\bigl( \begin{smallmatrix}}
\newcommand{\esm}{ \end{smallmatrix} \bigl)} 

\newcommand{\mbf}{\mathbf}

% ### Matrices and Vectors ###
\newcommand{\mtx}[1]{\ensuremath{\boldsymbol{#1}}}
\newcommand*\Let[2]{\State #1 $\gets$ #2}

% ### Sets ###
\newcommand{\set}[1]{\ensuremath{\mathcal{#1}}}

% ### Other ###

\newcommand{\transpose}{^{T}}
\newcommand{\inv}{^{-1}}
\newcommand{\pseudoinv}{^{+}}

% ### Hyphenation ###
\hyphenation{a-na-ly-sis}

% ############## End Macros ##############


% Title
\title{\fontfamily{phv}\selectfont{Causal Discovery methods for Effective Connectivity}}
\author{
  \textbf{R. Janssen} - \href{mailto:ramon.janssen@student.ru.nl}{ramon.janssen@student.ru.nl} \\
  \textbf{T. de Ruijter} - \href{mailto:t.deruijter@student.ru.nl}{t.deruijter@student.ru.nl}\\
  \textbf{T. Claassen} - \href{mailto:tomc@cs.ru.nl}{tomc@cs.ru.nl}\\
  \textbf{M. Hinne} - \href{mailto:mhinne@cs.ru.nl}{mhinne@cs.ru.nl}
}

\date{\fontfamily{ptm}\selectfont{\small{\bfseries{\today - Radboud
Universiteit Nijmegen}}}\\[0.5cm]\rule{\linewidth}{0.3mm}}

\begin{document}


\section{Causal discovery}
An approach for finding effective connectivity may be found in the domain of causal discovery.
A causal discovery method uses data about events to determine which of these events are correlated, and if possible, it determines which event causes which.
The definition of a causal relation is not trivial \cite{?} and the relevant principles will be covered in this section.
In our context, the causal discovery methods which have been used aim at inferring a Bayesian network.
This is done by calculating statistical (in)dependencies between stochastic variables.

Such a Bayesian network can be modelled as a graph with directed edges.
Each variable is represented by a node, and correlations between variables are represented by edges between those nodes.
A causal relation between variables $X$ and $Y$ is denoted with a directed edge; $X \rightarrow Y$ implies that $X$ is a cause of $Y$.
It is also said that $X$ is the parent of $Y$, and $Y$ is the child of $X$.
A Bayesian network cannot contain cycles, as that would imply that a variable is it's own (indirect) parent and thus it's own cause.
Ideally, a causal discovery methods finds only directed edges, resulting in a directed acyclic graph (DAG).
In practice, however, often not all causal relations can be inferred and undirected edges are also used in a graph.
As such, the result of a causal discovery method is a partially directed acyclic graph (PDAG).
Image {plaatje!} shows an example of a Bayesian network.
A causal discovery method attempts at finding a real-world Bayesian network.
But real-world situations like a brain are immensely complex and contain an unimaginable large number of atoms, which could all function as seperate variables.
As such, we represent our real-world, underlying Bayesian network with a much simpler model.
Such a model contains a reduced number of variables to be able to analyse and visualize the data.

An important principle for our approach is that of conditional dependence.
The dependency of two variables might depend on the value of a third variable (or possibly even more variables).
This principle is best demonstrated with an example.
When two variables, such as two switches being turned off or on, both cause a light bulb to glow, the two switches might not be dependent.
Whether one switch is turned on or off does not cause the other to switch on or off.
However, when the light bulb is glowing, the switches do become dependent.
If one switch is off, the other one is probably on.
Or in other words, one switch being turned on decreases the chance of the other switch being turned on.
In this case, we conclude that the switches are conditionally dependent given that the light bulb is turned on.
More generally, two variables $X$ and $Y$ can be conditionally dependent on a set of other variables $S$. % het juiste latex-symbooltje vinden :(

A similar principle is conditional independence. If two variables are 

Whether there is a dependency between two variables can be concluded by using a statistical test.
In this way, a graph can be inferred representing all dependencies between nodes.
However, this only results in an undirected graph, as it does not give any information about causality.
To derive causal relations, we can make use of conditional dependence.
Three rules for finding causality are used in the PC-algorithm.
One of those rules is based on conditional dependency 



%\bibliography{references}{}
%\addcontentsline{toc}{section}{References}
%\bibliographystyle{apalike}

\end{document}