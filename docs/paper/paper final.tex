% Needed packages
\documentclass[a4paper, 10pt, english, onecolumn]{article}
\usepackage[english]{babel}
\usepackage[cm]{fullpage}
\usepackage{cite}
\usepackage{anysize}
\usepackage{setspace}
%\usepackage[compact]{titlesec}
\usepackage{graphicx}
\usepackage{stfloats}
\usepackage{listings}
\usepackage{hyperref}

\usepackage{amssymb,amsmath}
\usepackage{algorithmicx}
%\usepackage{algorithmic}
\usepackage{algorithm}
\usepackage[noend]{algpseudocode}

% for coloring individual cells in a table
\usepackage[table]{xcolor}
%\usepackage{pgfgantt}

%\newcommand{\keywords}[1]{\par\noindent 
%{\bf Keywords\/}. #1}

% Margins & Headers
\marginsize{2.5cm}{2.5cm}{3.0cm}{2.0cm}
\columnsep 0.4in
\footskip 0.4in 
\usepackage{changepage}

% E-mail formatting
\usepackage{color,hyperref}
    \catcode`\_=11\relax
    \newcommand\email[1]{\_email #1\q_nil}
    \def\_email#1@#2\q_nil{
      \href{mailto:#1@#2}{{\emailfont #1\emailampersat #2}}
    }
    \newcommand\emailfont{\sffamily}
    \newcommand\emailampersat{{\color{red}\small@}}
    \catcode`\_=8\relax 
	
% List modifications
\newenvironment{packed_item}{
\begin{itemize}
  \setlength{\itemsep}{1pt}
  \setlength{\parskip}{0pt}
  \setlength{\parsep}{0pt}
}{\end{itemize}}

\newenvironment{packed_enum}{
\begin{enumerate}
  \setlength{\itemsep}{1pt}
  \setlength{\parskip}{0pt}
  \setlength{\parsep}{0pt}
}{\end{enumerate}}

% ### Mathematics ###
\newcommand{\bpm}{\begin{pmatrix}}
\newcommand{\epm}{\end{pmatrix}}

\newcommand{\bbm}{\begin{bmatrix}}
\newcommand{\ebm}{\end{bmatrix}}

\newcommand{\bsm}{\bigl( \begin{smallmatrix}}
\newcommand{\esm}{ \end{smallmatrix} \bigl)} 

\newcommand{\mbf}{\mathbf}

% ### Matrices and Vectors ###
\newcommand{\mtx}[1]{\ensuremath{\boldsymbol{#1}}}
\newcommand*\Let[2]{\State #1 $\gets$ #2}

% ### Sets ###
\newcommand{\set}[1]{\ensuremath{\mathcal{#1}}}

% ### Other ###

\newcommand{\transpose}{^{T}}
\newcommand{\inv}{^{-1}}
\newcommand{\pseudoinv}{^{+}}

% ### Hyphenation ###
\hyphenation{a-na-ly-sis}

% ### dots at the end of arrows and lines ###

\def \orightarrow {\circ\hspace{-0.42em}\rightarrow}
\def \oleftarrow {\leftarrow\hspace{-0.42em}\circ}
\def \orightline {\circ\hspace{-0.16em}-}
\def \oleftline {-\hspace{-0.16em}\circ}
\def \oline {\circ\hspace{-0.38em}-\hspace{-0.2em}\circ}
\def \srightarrow {\text{\textasteriskcentered}\hspace{-0.62em}\rightarrow}
\def \sleftarrow {\leftarrow\hspace{-0.62em}\text{\textasteriskcentered}}
\def \srightline {\text{\textasteriskcentered}\hspace{-0.16em}-}
\def \sleftline {-\hspace{-0.40em}\text{\textasteriskcentered}}
\def \sline {\text{\textasteriskcentered}\hspace{-0.38em}-\hspace{-0.4em}\text{\textasteriskcentered}}
\def \soline {\text{\textasteriskcentered}\hspace{-0.16em}-\hspace{-0.4em}\circ}
\def \osline {\circ\hspace{-0.38em}-\hspace{-0.16em}\text{\textasteriskcentered}}

% ############## End Macros ##############

% ### Section Formatting ###
\usepackage[compact]{titlesec}
\titlespacing*{\section}{4pt}{*0}{4pt}

% ### Paragraph Formatting ###
\makeatletter
\renewcommand{\paragraph}{%
  \@startsection{paragraph}{4}%
  {\z@}{0.5ex \@plus 1ex \@minus .2ex}{-1em}%
  {\normalfont\normalsize\bfseries}%
}
\makeatother

% Title
\title{\fontfamily{phv}\selectfont{Causal Discovery methods for Effective Connectivity in Human Brains}}
\author{
  \textbf{R. Janssen} - \href{mailto:ramon.janssen@student.ru.nl}{ramon.janssen@student.ru.nl} \\
  \textbf{T. de Ruijter} - \href{mailto:t.deruijter@student.ru.nl}{t.deruijter@student.ru.nl}\\
%  \textbf{T. Claassen} - \href{mailto:tomc@cs.ru.nl}{tomc@cs.ru.nl}\\
%  \textbf{M. Hinne} - \href{mailto:mhinne@cs.ru.nl}{mhinne@cs.ru.nl}
}

\date{\fontfamily{ptm}\selectfont{\small{\bfseries{\today - Radboud
Universiteit Nijmegen}}}\\[0.5cm]\rule{\linewidth}{0.3mm}}

\begin{document}

\maketitle

\setlength{\parindent}{0.0cm}
\setlength{\parskip}{3mm plus2mm minus1.5mm}

\begin{abstract}
% TODO: first state what you try to do, then how you did it, then what it brought you
%In this work we applied PC algorithm on resting-state data of healthy human subjects.
%Additionally we propose a variant of PC algorithm that poses weaker assumptions on the model, better fitting the underlying model.
%We find that detecting structure homological areas in the brain works better with PC algorithm than most conventional diffusion weighted imaging techniques.
%In the area of directionality and causality, more research is needed.
\end{abstract}
\section{Introduction}
\subsection{Subject introduction}
%Effective connectivity resting state analysis in human brains
%Resting state
%Brains are networks; apply graphical models and causal discovery

\subsection{ Motivation for research}
%Better understanding of brains
%Relatively new field - not yet an established analysis method
%Existing problems with current methods and applications, specifically [Tom: mention all, or just the relevant problems?]
%- Model selection
%- Indirect measurements
%- Modelling causal structure across individuals (varying responses to same stimuli)
%- Distinct but overlapping ROIs
%- Varying Haemodynamic response (HRF)
%Find shortcomings of PC and improve it (EMS-PC)

\subsection{Problem statement}
%See whether applying causal methods (PC and EMS-PC) can help bring new insights and whether we can find evidence for the existence of causal patterns on a whole brain level.

\subsection{Previous research}
%[Tom: we did not explicitly have this paragraph, so this is additional effort]
%[Ramon]: Lijkt me vergelijkbaar met de verwijzingen naar literatuur die we al hebben? Met wat extra referenties naar wat Max zei: temporele methodes zijn onbetrouwbaar, dat hebben wij niet
%[Ramon]: even een goede balans zoeken tussen dit en "Existing problems" in motivation

\subsection{Research goal}
%How does the standard causal discovery method PC-Algorithm behave regarding the aforementioned problems when applied to resting-state functional time-series data?
%- Can we overcome the consistency problems of PC-algorithm applied on resting-state functional data?
%- Does the PC algorithm applied on resting-state functional data provide consistent, specific and anatomically plausible connectivity?
%- How does functional connectivity found by the first part of PC relate to other structural / functional methods applied on the same resting-state functional data regarding the aforementioned problems?
%
%How does the performance of EMS-PC relate to PC regarding the aforementioned problems when applied to the same resting-state functional time-series data?
%- When would EMS-PC perform better than PC-algorithm regarding directionality results?
%- Is EMS-PC more consistent in its results within a single subject than PC-algorithm on resting-state functional data, regarding the multiple separating set addition and the explicit v-structure test?
%- Can we find unfaithfulness within our model without additional independence tests?


\subsection{Relevance of research}
%If HRF can be avoided, causal patterns can be established with greater reliability
%Insight and diagnostics in neuro-degenerative diseases
%Any found evidence of causal patterns can give an impulse to further research
%[Ramon: is dit niet al duidelijk uit motivation? Lijkt me dubbelop]

\section{Background}
\subsection{Types of brain connectivity}% [optional] [Ramon: lijkt me wel belangijk, geeft ook het belang aan van richting zoeken, hoeft niet lang]
%Functional
%Structural
%Effective

\subsection{Causal discovery theory}
%Causal Markov Condition
%Causal faithfulness
%d-separation
%V-structure
%Causal sufficiency
%Latent variables

\subsection{PC algorithm}
%Algorithm, assumptions and explanation [Ramon: standaard PC een naam geven en consistent gebruiken, bijv. standard PC?]
\paragraph{Description}
The PC algorithm is a so-called constraint based method, where solutions are found by incrementally leaving out or adding parts that do not violate a set of constraints.
Starting from a fully connected undirected graph, the PC algorithm forms structure by cutting away edges between conditionally independent variables.
In the second step, orientation rules are applied based on found separating sets and structure.
The original PC algorithm assumes causal sufficiency, excluding the existence of common confounders between nodes.
This seems unwanted when handling brain data, where one would expect a large amount of latent processes.
Also measuring methods are not yet advanced enough to measure detailed neuronal processing of the brain, thus skipping over details.
In the perspective of causal inference, these missing areas can also appear as confounders.
For the following experiments we have used the causal insufficient version of PC algorithm, modified to handle latent variable as described by Spirtes, Glymour and Scheines \cite[p.165-167]{spirtes2000}.
From now on, we will refer to this algorithm as \textit{standard PC}.
Pseudocode for this method is shown in algorithm \ref{alg:pc_alg}.

\begin{algorithm}
\caption{Standard PC algorithm}
\begin{spacing}{1.1}
\begin{algorithmic}[1]
\Function{PC}{$data$}
\State connect all pairs of point in undirected graph $G$. \label{alg:pc_alg:p1_start}
                                                                                                                                    \Comment{Structural Part} 
\State $n \gets 0$
\While{there are connected pairs ($X$, $Y$) such that $n < |\operatorname{Adjacencies}(X)\setminus\{Y\}| $}
  \ForAll{such connected pairs ($X$, $Y$)}
    \ForAll{subsets $S$ of $\operatorname{Adjacencies(X)}$ for which $|S| = n$}
      \If{$X$ and $Y$ are independent given $S$}
        \State remove edge $X - Y$ from G
        \State record $S$ as separating set of ($X, Y$)
        \State continue with the next pair ($X$, $Y$) \label{alg:pc_alg:next_pair}
      \EndIf
    \EndFor
  \EndFor
  \State $n \gets n+1$ \label{alg:pc_alg:p1_end}
\EndWhile
\\
\State $\text{PDAG} \gets G\text{, with all connections replaced by }\oline$ \label{alg:pc_alg:p2_start}
                                                                                                                                    \Comment{Directional Part}
\ForAll{unshielded triples $\left <X, Y,Z \right> $ in $G$ }
                                                                                                                                    \Comment{Find V-structures}
  \If{$Y$ is not in the separating set of $X$ and $Z$} \label{alg:pc_alg:v_struct}
    \State Orient $X \sline Y \sline Z$ as $X \srightarrow Y \sleftarrow Z$ in $\text{PDAG}$
  \EndIf
\EndFor
\While{Edges can still be oriented}
                                                                                                                                    \Comment{Additional orientations}
  \If{there is an edge $X \sline Y$, and there is a directed path from $X$ to $Y$}
    \State orient $X \sline Y$ as $X \rightarrow Y$
  \EndIf
  \If{there are edges $X \srightarrow Y$ and $Y \sleftline Z$, the latter is not an edge $Y \sleftarrow Z$}
    \If{$X$ and $Z$ are not connected}
      \State orient $Y \sline Z$ as $Y \rightarrow Z$
    \EndIf
  \EndIf
\EndWhile
\State \Return $G$ and the separating sets \label{alg:pc_alg:p2_end}
\EndFunction
\end{algorithmic}
\end{spacing}
\label{alg:pc_alg}
\end{algorithm}

%Before finding directed edges, the PC-algorithm first finds undirected structure.
%For each two non-adjacent nodes $X$ and $Y$ standard PC finds a minimal separating set, which is the smallest set acting as a separating set.
%This is done by first assuming all nodes are connected, and then testing all pairs of nodes for independence.

Starting from a fully connected graph, standard PC finds a minimal separating set for every pair of non-adjacent nodes $X$ and $Y$.
These are lines \ref{alg:pc_alg:p1_start} to \ref{alg:pc_alg:p1_end} in the pseudocode.
The independence of $X$ and $Y$ is tested given a subset of neighbours of $X$ or $Y$ using a statistical independence test such as the Chi-square or Fisher-Z independence tests. %TODO: Add citation to these methods.
If such an independence is found, this subset is a separating set of $X$ and $Y$ and they are not connected.
If no such set can be found, the variables are assumed to be dependent and thus remain connected.
To find a minimal separating set, all subsets of neighbours are traversed in order of size; the first separating set which is found, is the minimal one.

Next, PC algorithm attempts to orient these edges.
These are lines \ref{alg:pc_alg:p2_start} to \ref{alg:pc_alg:p2_end} in the pseudocode.
The 'o' mark represents that it is unknown what orientation the specific edge should have thus representing an entire Markov class of solutions rather than a specific instance.
Like Spirtes et al, we use a star-symbol at the end of an edge to denote that any mark may be present: an arrowhead, an "o" or an empty mark.
When an edge is oriented as an edge with a star, this denotes that this end is not changed.
The first of the directionality rules attempts to orient unshielded triples.
For a triple $\left < X,Y,Z, \right>$ to be oriented, $X$ and $Z$ should have a separating set $S$ not containing $Y$.
If so, the triple is marked a V-structure.
After all V-structures have been found, the other rules are applied in no particular order.
These rules are based on already found directionality.

\paragraph{Run-time performance}
The PC-algorithm run-time is in the order of the amount of dependency tests to be performed and thus it has a worst-case runtime complexity exponential in the maximal branching degree present in the graph.
However, this worst-case scenario only happens when no edges can be removed; all subsets of the complete set of nodes have to be used to test d-separation of every pair of nodes.
For a sparse graph, many edges may be removed for small separating sets, and this enables the standard PC algorithm to often run fast in practice perform well on hundreds of variables. % TODO: add citation to example publication. Peter Lucas maybe or examples in the sprites book?

\paragraph{Shortcomings}
Although PC is proven correct, the version provided above as standard PC is not.
It is also not complete.
There is additional the number of possible graphs in the returned Markov class increasing exponentially with the amount of variables in the provided data, thus not always providing ample information.
Also, standard PC is dependent on the order in which these variables are presented, as this is the same order in which edges between pairs of these variables are considered for removal\cite[p.88]{spirtes2000}.

In practice, the lack of full correctness and completeness are of little concern as it involves only very specific and little occurring connectivity patterns\cite[p.127-130]{spirtes2000}.
% TODO: Paragraph seems incomplete.

\paragraph{Conservative PC extension}
The V-structure rule in PC algorithm orients unshielded triples based on a single separating set. 
Simply put, it is assumed the single found separating set is enough to show conditional independence. 
Ramsey et al. \cite{ramsey2012} introduce a weaker but still sufficient assumption for finding V-structures.
Through this weaker but still adjacency-faithful assumption, the method is more cautious than PC algorithm in drawing unambiguous conclusions on causal orientations.
That is why this method is named Conservative PC (CPC).
The resulting algorithmic changes provide better results and is able to mark incorrectly unshielded triples as unfaithful.
Sadly, CPC requires a significant amount of additional conditional independence tests, exponential in the largest branching degree of the tested structure.
Although the weaker assumptions made with conservative PC might be useful for this research, the run-time performance is not good enough to apply it on our dataset.

\section{Methods}% (our contributions) [Tom: I'm not completely satisfied with this structure. It's too split and diverse to put everything under 'methods']
\subsection{Improvements to PC}
%Motivation (refer to results-section)
%Multiple separating sets and why it should be better / not worse (correctness, runtime)
%Partial unfaithfulness check and why it should be better / not worse (correctness, runtime)
%EMS-PC extension (no pseudo-code, but additions regarding standard PC)

\paragraph{Multiple Separating Sets}
As will be explained in the results section, standard PC is a bit hasty in assigning v-structures, resulting in many unnecessary and perhaps unfaithful directionality.
The idea of CPC, to make a weaker assumption on graph faithfulness than standard PC is potential as less erroneous arrows are found.
Instead of doing additional tests as described in CPC, one could attempt to determine causal unfaithfulness with less dependency tests.
An adaptation to PC algorithm that would not increase the order of complexity is once a separating set is found to test all other vertex subsets of the same order as well.
This finds all separating sets of the lowest order without increasing complexity.
Orienting an unshielded triple $\left<A,B,C\right>$ requires $B$ not to occur in any separating set of $(A,C)$.
By having more than one separating set, this constraint is strengthened is the data is not faithful.
If it is, the results are not changed.
This also does not change the run-time complexity.

% Unfaithfulness test
One could use these additional separating sets to test oriented unshielded triples for unfaithfulness, similar to CPC.
Apart from serving as a sanity check, this could serve as the basis of a repair algorithm for errors made by PC.

% Explicit test
A more explicit way to decrease faulty orientation of an unshielded triple $\left<A,B,C\right>$ is to check whether adding $B$ to a separating set of $(A,C)$ makes the pair conditionally dependent.
If so, the graph might become more faithful to its modeled distribution; If standard PC might orient a V-structure, whereas this condition does not hold, then the concluded V-structure might be erroneous.
For faithful graphs, this extra condition should not change the result, as it will always hold.

\paragraph{EMS-PC}
We have combined the above ideas - multiple separating sets, unfaithfulness check and explicit orientation check - into an adaptation of PC, which we refer to as Explicit Multiple Separating Set PC (EMS-PC).
These three changes have been made in this adaption with respect to standard PC as described in algorithm \ref{alg:pc_alg}.
At line \ref{alg:pc_alg:next_pair}, the search for separating sets for $X$ and $Y$ stops when a separating set is found; instead, EMS continues the search for more separating sets of the same order.
All of these separating sets are recorded.

Also, the condition on line \ref{alg:pc_alg:v_struct} to orient $X - Y - Z$ according to the V-structure-rule has been made more strict.
As multiple separating sets have been found instead of only one, it needs to be checked whether $Y$ is in not in \emph{any} of those sets before the V-structure can be oriented.
In addition to the original condition, $X$ and $Z$ are tested for \textit{d}-separation given $\{Y\}$.
If they are found to be conditionally dependent, the triple is oriented as as $X \srightarrow Y \sleftarrow Z$.
%As a final safe-guard, all unshielded triples are checked for unfaithfulness and marked as such if applicable using found separating sets. %TODO wat bedoel je hiermee?
%TODO Willen we nog iets over unfaithful dingen zeggen? Zo ja, dan wel een antwoord op deze vragen...
% TODO: How do you mark unfaithful triples? can you spot the impact in the output?


\subsection{Experimental validation}
\paragraph{Experimental pipeline}
%- What data is needed
%- Apply structure PC on resting-state functional time-series
%- Apply directional PC
%- Parameter optimisation (model selection)

\paragraph{Standard PC}
%- Measuring whole-brain structural and directional consistency of PC - frequency of occurrences within subjects
%- Markov equivalence class size (nr. of latent variables / bidirectional arrows) within subjects
%- Markov class Asymmetry and certain asymmetry within subjects
%- Consistency of structure across subjects when aggregating results (averaging and frequency of occurences)
%- Consistency of directionality across subjects when aggregating results

\paragraph{EMS-PC}
%- Measuring whole-brain structural and directional consistency of EMS-PC
%- Markov equivalence class size (nr. of latent variables / bidirectional arrows) within subjects
%- Markov class Asymmetry and certain asymmetry within subjects
%- Amount of separating sets found
%- Unfaithfulness check with and without additional separating sets

\paragraph{Comparison}
%- Comparison of directionality between standard PC and EMS-PC on consistency of results within subjects
%- Comparison of ... of results across subjects
%- Run-time of methods and comparison between run-times
%- Comparison of structure (functional connectivity) of PC (both. they're identical in structure) with DWI results by Max et al.

\subsection{Data acquisition}
%Subject acquisition and instruction
%Used equipment and specs
%Experimental setup
%Scanner settings
%Preprocessing
%- Software
%- Scripts
%Resulting data specs
%[Ramon: tot hoeverre is dit in detail nodig? Enerzijds moeten we niet doen alsof het kant en klaar uit dat apparaat komt, maar anderzijds wordt het in Max' artikel beschreven. Zijn dingen als "settings" dan niet overkill?]

\section{Results}\label{sec:results}
%Answer all the questions as mentioned in methods, in the same order
%No aggregation over direction, motivate this

\section{Discussion}
%Poor consistency between subjects
%Good consistency of structure within and between subjects
%inter-hemisphere connection
%cause of poor consistency: inherent to our method or viable for further research?
%data aquisition and noise, improvement in data -> improvement in causal inference
%better methods for direction (Tom's new algorithm), and why that might solve causal inference problems
%run-time is quick
%HRF

\subsection{Conclusion}
%repeat everything in 1 paragraph, or leave away, whatever seems most fitting in the end. HRF makes us awesome


\bibliography{references}{}
\addcontentsline{toc}{section}{References}
\bibliographystyle{plain}

\end{document}