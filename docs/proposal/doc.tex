% Needed packages
\documentclass[a4paper, 10pt, english, twocolumn]{article}
\usepackage[english]{babel}
\usepackage[cm]{fullpage}
\usepackage{cite}
\usepackage{anysize}
\usepackage[compact]{titlesec}
\usepackage{graphicx}
\usepackage{stfloats}
\usepackage{listings}
\usepackage{hyperref}

\usepackage{amssymb,amsmath}
\usepackage{algorithmicx}
\usepackage{algorithm}
\usepackage[noend]{algpseudocode}

% for coloring individual cells in a table
\usepackage[table]{xcolor}
%\usepackage{pgfgantt}

\newcommand{\keywords}[1]{\par\noindent 
{\bf Keywords\/}. #1}

% Margins & Headers
\marginsize{2.5cm}{2.5cm}{2cm}{1.5cm}
\columnsep 0.4in
\footskip 0.4in 
\usepackage{changepage}

% E-mail formatting
\usepackage{color,hyperref}
    \catcode`\_=11\relax
    \newcommand\email[1]{\_email #1\q_nil}
    \def\_email#1@#2\q_nil{
      \href{mailto:#1@#2}{{\emailfont #1\emailampersat #2}}
    }
    \newcommand\emailfont{\sffamily}
    \newcommand\emailampersat{{\color{red}\small@}}
    \catcode`\_=8\relax 
	
% List modifications
\newenvironment{packed_item}{
\begin{itemize}
  \setlength{\itemsep}{1pt}
  \setlength{\parskip}{0pt}
  \setlength{\parsep}{0pt}
}{\end{itemize}}

\newenvironment{packed_enum}{
\begin{enumerate}
  \setlength{\itemsep}{1pt}
  \setlength{\parskip}{0pt}
  \setlength{\parsep}{0pt}
}{\end{enumerate}}

% ### Mathematics ###
\newcommand{\bpm}{\begin{pmatrix}}
\newcommand{\epm}{\end{pmatrix}}

\newcommand{\bbm}{\begin{bmatrix}}
\newcommand{\ebm}{\end{bmatrix}}

\newcommand{\bsm}{\bigl( \begin{smallmatrix}}
\newcommand{\esm}{ \end{smallmatrix} \bigl)} 

\newcommand{\mbf}{\mathbf}

% ### Matrices and Vectors ###
\newcommand{\mtx}[1]{\ensuremath{\boldsymbol{#1}}}
\newcommand*\Let[2]{\State #1 $\gets$ #2}

% ### Sets ###
\newcommand{\set}[1]{\ensuremath{\mathcal{#1}}}

% ### Other ###

\newcommand{\transpose}{^{T}}
\newcommand{\inv}{^{-1}}
\newcommand{\pseudoinv}{^{+}}

% ### Hyphenation ###
\hyphenation{a-na-ly-sis}

% ############## End Macros ##############


% Title
\title{\fontfamily{phv}\selectfont{Causal Discovery methods for Effective Connectivity}}
\author{
  \textbf{R. Janssen} - \href{mailto:ramon.janssen@student.ru.nl}{ramon.janssen@student.ru.nl} \\
  \textbf{T. de Ruijter} - \href{mailto:t.deruijter@student.ru.nl}{t.deruijter@student.ru.nl}\\
  \textbf{T. Claassen} - \href{mailto:tomc@cs.ru.nl}{tomc@cs.ru.nl}\\
  \textbf{M. Hinne} - \href{mailto:mhinne@cs.ru.nl}{mhinne@cs.ru.nl}
}

\date{\fontfamily{ptm}\selectfont{\small{\bfseries{\today - Radboud
Universiteit Nijmegen}}}\\[0.5cm]\rule{\linewidth}{0.3mm}}

\begin{document}

\maketitle

\begin{abstract}
In the field of brain research, different forms of connectivity are considered. To find effective connectivity, many different approaches have been used. In this project, we will apply causal discovery the methods of PC and BCCD to structural and functional connectivity, to infer effective connectivity. As data from multiple subjects will be used, evaluation of the result can show whether the method yields consistent results.
\keywords{Keywords}
\end{abstract}

\setlength{\parindent}{0.0cm}
\setlength{\parskip}{0.25cm}

\section{Introduction}

% vergelijking met granger causality, en andere methodes?
% granger (Alard Roebroeck, Elia Formisano, Rainer Goebel, 2005) -> geen structural connectivity nodig (schaalt niet goed)
% covariance structural equation modeling -> structural connectivity nodig (Roebroeck, Formisano, Goebel, 2005)
% Multivariate autoregressive modeling (Harrison, Penny, Friston) -> Set of interaccting regions chosen beforehand (Roebroeck, Formisano, Goebel, 2005)
% Dynamic causal modelling (Friston, Harrison, Penny, 2003) -> Set of interaccting regions chosen beforehand (Roebroeck, Formisano, Goebel, 2005)

\paragraph{Motivation}
The brain and particularly the human brain have been studied for hundreds of years.
Today, the secrets of our brains still are one of the most sought-after.
The techniques however have changed.
We have come a long way since the time of Phrenology, the pseudoscience based attempting to derive cognitive ability and personality from measurements of the skull or, post-mortem, the brain.
Advanced measurement methods exist that allow scientists to peek inside brains that give a coarse but broad overview without opening a single skull.

Of particular interest are causal relations between brain regions and cognitive and bodily functions.
The question whether brain region X is somehow connected to brain region Y is still highly relevant; it is the main question of the present-day scientific field of brain connectivity.
Understanding brain structure implies understanding more about the brain as a whole.

Not surprisingly, brain connectivity also finds applications in medicine.
Neuro-degenerative diseases such as Alzheimer's disease, Parkinson's disease, dementia, Amyotrophic lateral sclerosis (ALS) have all been shown to severely alter brain connectivity. %TODO: Citation needed.
Better methods for analysing connectivity could lead to more insight in these diseases.

The field of brain connectivity finds its roots in the early 1990s \cite{friston1993functional, friston1994}, though because of more recent developments in the field of Artificial Intelligence and Machine Learning it is possible to do more thorough analysis \cite{vandenheuvel2010}.
One type of connectivity analysis involves finding causal relations between brain areas, specifically \emph{effective connectivity}; How does the activity of one region cause the activity of another?
However, causal analysis is also relatively new and its application in brain connectivity analysis still faces many challenges that require further research \cite{ramsey2010}.

\paragraph{Problem statement}
This project seeks to utilise the strengths of current state-of-the-art causal discovery methods by applying them to resting state brain fMRI data in order to find new applicable methods for determining causal patterns in the human brain.

The work by \cite{ramsey2010} addresses several problems causal analysis in fMRI analysis suffers from.
Methods in causal discovery have high computational costs, making it nearly impossible to calculate networks with more than hundreds of nodes.
Another important fact is that fMRI Analysis indirectly measures brain activity.
Brain models that do not account for possible latent - indirect - sources within the brain or the shortcomings of fMRI may suffer from noise or fail to capture underlying patterns.
Solutions to this particular problem have been proposed that do account for latent sources \cite{ramsey2010, waldorp2011}.
Another problem is the strong diversity between subjects.
Every brain is inherently different, making it non-trivial to combine subject data or even draw conclusions across subjects.
Even within a single brain it changes how different areas react over time.

Several methods for finding effective connectivity already exist \cite{mclntosh1994, harrison2003, friston2003, roebroeck2005}.
However, using more generally applicable causal discovery methods has not been elaborated on much.
Also, none of these provide a measure of uncertainty.
It is a fact that there are errors in any graph produced.
This is inherent to the methods and the coarseness of the measurements.
As there is no standard baseline to compare results with, a measure of uncertainty would be highly preferable.
A framework for such a probabilistic approach is introduced in \cite{claassen2012}.

We would like to know whether applying such methods, such as the approach introduced in \cite{claassen2012}, or a standard approach such as the PC-algorithm \cite{spirtes2000} can solve some of the difficulties effective connectivity suffers from.

% Structure of the rest of proposal
In the remainder of this proposal we briefly discuss some necessary background knowledge on connectivity and methods, we then discuss our strategy and finally we present a time-plan overview.

\section{Theory}

\subsection*{Brain connectivity}

%Question: should this be included?
\subsubsection*{Functional connectivity}
Functional connectivity describes the statistical dependence of neuronal activity between different brain regions \cite{friston1993functional}.
This gives an insight in the organisation of the brain.
Functional connectivity is strongly time-dependent, as activity changes rapidly providing only a low emporal resultion.
This means measured dependencies can be the result of statistical noise.
Functional connectivity can be deduced by fMRI measurement of brains in resting-state \cite{Lowe2000, doria2010, Bullmore2009}.
Resting state indicates that subjects are instructed to relax without thinking of anything in particular to stimulate spontaneous brain activity.

\subsubsection*{Structural connectivity}
Structural connectivity is defined as the mapping of anatomical - neural - paths in the brain between different brain regions \cite{friston1994}.
This is strongly related with functional connectivity: regions can only be functionally connected if there is a structural relation between them \cite{cabral2012}.
This intuitive relation can be demonstrated empirically \cite{vandenheuvel2009}.
Structural connectivity is less time dependent as it involves mappings of anatomical connections opposing temporal activity patterns.

\subsubsection*{Effective connectivity}
In contrast to functional and structural connectivity, effective connectivity takes into account the cause and effect of relations.
It has been described as ``the influence one neural system exerts over another'' \cite{friston1994}.
Effective connectivity indicates which brain regions stimulate other regions.
Time series need to be analysed to infer effective connectivity as cause and effect can be deduced from which event precedes which. 
It is possible to infer effective connectivity from structural and functional connectivity \cite{mclntosh1994, harrison2003, friston2003, roebroeck2005}.

Research has been done on inferring effective connectivity directly with several methods.
The most renowned methods are Granger causality \cite{roebroeck2005}, Structural Equation Modelling \cite{mclntosh1994}, Multivariate Autoregressive Modeling \cite{harrison2003} and Dynamic Causal Modelling \cite{friston2003}.
These are all methods dedicated to this problem.
These methods also suffer from similar problems of which perhaps the most important is scaling.
Most of these methods can only handle tens of nodes in reasonable time, which becomes problematic in realistic situations with up to tens of thousands of nodes, ideally.

\subsection*{Causal discovery methods}
The first causal method we will apply on brain structure data will be the PC-algorithm \cite{spirtes2000}.
It is used to find structure, as well as causal relations between vertices in directed graphs. 
If we already provide a graph skeleton, PC-algorithm can be used to determine directions.
An adaptation of PC that is better adapted to finding latent variables is Fast Causal Inference \cite{spirtes2000}, though this method does not scale well to tens of variables.

Another causal discovery method is the Baysian Constraint-based Causal Discovery (BCCD) algorithm \cite{claassen2012}, which gives similar results as PC, and is designed in an attempt to improve PC.
This algorithm has some useful properties; it is a Bayesian approach and as such, it is more robust against noise.
The authors have shown the method to perform as good as or better than other causal discovery algorithms such as PC on representative datasets.
As is inherent to Bayesian methods, or approximations of Bayesian methods, the resulting network graph is not a  single structure but rather a distribution over possible structures.
Concretely, this provides an uncertainty measure over edges in the network, giving additional information compared to PC.

\begin{figure*}[bp]
\begin{tabular}{c || c | c | c | c | c | c | c | c | c | c |}
\setlength{\tabcolsep}{12pt}
\renewcommand{\arraystretch}{1.3}
Week          & 41 & 42 &  43 &  44 &  45 &  46 &  47 &  48 &  49 &  50 \tabularnewline \hline
PC            & \multicolumn{2}{ |c| }{\cellcolor[gray]{0.9} Make implementation} & \multicolumn{2}{ |c| }{\cellcolor[gray]{0.9} Structural}  & \cellcolor[gray]{0.9} Functional & \multicolumn{4}{ |c| }{} & \tabularnewline \hline
BCCD          & \multicolumn{4}{ |c| }{} & \multicolumn{2}{ |c| }{\cellcolor[gray]{0.85} Structural} & \cellcolor[gray]{0.85} Functional & \multicolumn{3}{ |c| }{} \tabularnewline \hline
Analysis      & \multicolumn{3}{ |c| }{} & \multicolumn{2}{ |c| }{\cellcolor[gray]{0.9} PC} & & \multicolumn{2}{ |c| }{\cellcolor[gray]{0.9} BCCD} & \multicolumn{2}{ |c| }{} \tabularnewline \hline
Write article & \multicolumn{3}{ |c| }{} & \multicolumn{7}{ |c| }{\cellcolor[gray]{0.85}}  \tabularnewline \hline
\end{tabular}
\caption{Project time table. `Structural' implies applying the given method using structural data as a network skeleton. `Functional' implies inferring structure with the given method first through functional data.}
\end{figure*}

\subsection*{Data acquisition \& analysis}
We will apply the mentioned methods on structural and functional data gathered from human subjects.
We use the same functional and structural data as in Hinne et al. \cite{hinne2013}, of which six subjects were selected.
To find effective connectivity, both are needed.
Structural connectivity of these subjects can be derived from DTI-measurements, using probabilistic tractography \cite{hinne2013}.
Functional connectivity can be extracted from fMRI-measurement, possibly in combination with the structural 
connectivity.
Functional connectivity can be extracted from fMRI-measurement, possibly in combination with the structural connectivity.


\section{Methods \& Strategy}
%Short general introduction to both methods
We use an existing implementation of BCCD to infer effective connectivity; PC will be implemented in this project.
An implementation of BCCD suitable for this project is made available by the department of Intelligent Systems at the Radboud University Nijmegen.

The first step will be to get the implementations for the algorithms working on the datasets with functional and structural data.

Firstly we will use structural networks as a basis for the algorithms, and use functional information to infer causal relations.
Later we will use functional information only, to infer both the network and the causal relations.
We will compare this with the previous.
As PC is a well-described algorithm \cite{spirtes2000} and should not be difficult to implement, so this will be done within this project.

The second step is to apply the BCCD-algorithm, also to be used with and without provided network structure.
Both methods provide a connectivity matrix representing effective connectivity.

Results will then be evaluated and compared over all six subjects to see to which extent obtained networks are consistent across subjects.
If so, we will analyse the properties and difficulties presented by Ramsey et al. \cite{ramsey2010}.
We will conclude whether standard causal discovery methods provide a good direction of research for deriving and evaluating effective connectivity.

% Licht hier toe met welke methoden we gaan vergelijken en wat we nog meer besproken hebben in de outline.
% - Difficulties in causal discovery
% - Comparison methods
% - Comparison procedure
% - Properties

\section{Time schedule}
This project is divided into four main parts: applying the PC-algorithm, applying the BCCD-algorithm, analysing the data (including cross-subject comparison) and writing the article. The schedule for applying the algorithm has been subdivided into applying it with functional connectivity using structural data as a basis, and applying it to only functional data.



\bibliography{references}{}
\addcontentsline{toc}{section}{References}
\bibliographystyle{apalike}

\end{document}